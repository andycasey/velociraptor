\documentclass[twocolumn]{aastex61}
\usepackage{bm}
\usepackage{color}

\newcommand\teff{T_{\rm eff}}
\newcommand\logg{\log{g}}
\newcommand\feh{[\rm{Fe}/\rm{H}]}
\newcommand\mh{[\rm{M}/\rm{H}]}
\newcommand{\luminosity}{L_\circ}
\newcommand{\radius}{R_\circ}


\newcommand{\project}[1]{\textsl{#1}}
\newcommand{\package}[1]{\texttt{#1}}
\newcommand{\acronym}[1]{{\small{#1}}}
\newcommand{\Gaia}{\project{Gaia}}
\newcommand{\gaia}{\project{gaia}}
\newcommand{\todo}[1]{\textcolor{red}{#1}}
\newcommand{\rp}{\textsl{rp}}
\newcommand{\bp}{\textsl{bp}}


\newcommand{\NumberOfHigherOrderSystems}{XXX}
\newcommand{\NumberOfSingleLinedSystems}{XXX}


\newcommand{\GaiaRVE}{\sigma_{\mathrm{V}_\mathrm{R}}^\mathrm{MTA}}
\newcommand{\RVJitter}{\sigma(\mathrm{V}_\mathrm{R}^{t})}

\received{2018 XX XX}
\revised{2018 XX XX}
\accepted{2018 XX XX}


\submitjournal{AAS Journals}

\shorttitle{Stellar multiplicity}
\shortauthors{Casey et al.}

\begin{document}

\title{Detection and characterisation of stellar multiplicity from \Gaia\ data}

\correspondingauthor{Andrew R. Casey}
\email{andrew.casey@monash.edu}

\author[0000-0003-0174-0564]{Andrew R. Casey}
\affiliation{School of Physics \& Astronomy, 
			 Monash University,
			 Wellington Rd, Clayton 3800, Victoria, Australia}
\affiliation{Faculty of Information Technology, 
			 Monash University, 
			 Wellington Rd, Clayton 3800, Victoria, Australia}

\author[0000-0002-9328-5652]{Daniel Foreman-Mackey}
\affiliation{Flatiron Institute, 
			 162 Fifth Ave, New York, NY 10010, USA}

\author[0000-0003-2866-9403]{David W. Hogg}
\affiliation{Center for Cosmology and Particle Physics, Department of Physics,
    		 New York University, 
		  	 726 Broadway, New York, NY 10003, USA}
\affiliation{Center for Data Science, 
			 New York University, 60 Fifth Ave, New York, NY 10011, USA}
\affiliation{Max-Planck-Institut f\"ur Astronomie, 
			 K\"onigstuhl 17, D-69117 Heidelberg}
\affiliation{Flatiron Institute,
			 162 Fifth Ave, New York, NY 10010, USA}

\begin{abstract}
Most stars are in binary systems. The occurrence rate and properties of stellar 
multiples (e.g., binaries, triples) is relevant to many fields of astrophysics, 
but the effects of stellar multiplicity are commonly ignored in the analysis of 
most stellar and galactic surveys. Here we use \Gaia\ data to identify 
\todo{\NumberOfHigherOrderSystems} stellar multiple systems from radial velocity
and astrometric jitter, photometry, and intrinsic luminosity. Our sample includes
\todo{\NumberOfSingleLinedSystems} unresolved binary systems detected solely
from radial velocity or astrometric jitter, which have orbital periods up to 
$\approx$3.5\,yrs. We show that the ratio of jitter in radial velocity and
astrometry can be calibrated to estimate inclination angles (and companion masses)
for thousands of systems. \todo{We comment on stellar multiplicity with stellar
metallicity}. \todo{We comment on systems with stellar remnants (e.g. NS/BHs).}
\end{abstract}


\keywords{stars: dynamics}

\section{Introduction} \label{sec:intro}

Higher order stellar systems describes any stars that have at least one stellar 
companion (e.g., binaries, trinaries). The presence of stellar companions
introduces a number of complications to the inference of stellar and galactic
properties, but these factors are often ignored.

The \Gaia\ space telescope provides exquisite astrometry, photometry, and radial
velocity measurements over many years for million of point stars in our galaxy.
%These data provides an excellent opportunity to self-calibrate the errors in those
%measurements. 
When a point source has an unresolved stellar companion within some range of orbital
and stellar parameters, it is expected that there will be an anomalous excess in 
these measured properties relative to a single star of a similar type. This is
nothing new: astronomers have used radial velocity \citep{firstplanet} and 
astrometric variations \citep{someone} and photometry \citep{someone} to infer the 
presence of stellar and exoplanet companions forever. However, the opportunity
presented by \Gaia\ is to infer the presence of stellar companions for millions
of point sources, across both hemispheres. No ground- or space-based instrument
has ever provided such an opportunity.

In this work we make use of photometry and astrometry (including radial motion)
measurements in the second \Gaia\ data release to infer the presence of stellar
companions for millions of point sources, and provide some basic characterisations
of these systems. In Section \ref{sec:method} we describe our methods, and in
Section \ref{sec:results} we outline our results in context of other stellar
multiplicity surveys. We include comparisons of binary properties between
clusters and the field, as a function of stellar properties (e.g., stellar
metallicity), and highlight some particularly noteworthy systems that would
benefit from immediate follow-up spectroscopic observations. 


\section{Method}

There are many ways that binaries can be reliably identified using \Gaia\
data, including radial velocity measurements, photometry, and astrometry.
These detection methods are sensitive to binary systems with different 
properties. Here we describe the radial velocity information that forms 
part of our model before describing how photometry and astrometry are 
incorporated into the detection and characterisation of stellar multiples.

\subsection{SB1 Systems: Single-lined spectroscopic binaries}
\label{sec:sb1_methods}

Radial velocity measurements are not available for individual epochs in the 
second \Gaia\ data release \citep{Lindegren:2018,Gaia_RVS}, but the a radial velocity point estimate and error
is available for 7,224,631 sources \cite{gaia_rvs}. 
The radial velocity error provided by \Gaia\ ($\GaiaRVE$; column name \texttt{radial\_velocity\_error}) 
is a function of the number of transits $N$ (i.e., the number of observations; 
represented by column \texttt{rv\_nb\_transits}) and the scatter among those $N$ measurements. 
This allows us to calculate the scatter among those measurements, which we will
refer to as the \emph{radial velocity jitter}, for each source
\begin{eqnarray}
\RVJitter = \sqrt{\frac{2N}{\pi}}\GaiaRVE \quad .
\end{eqnarray}



For a single star (without a significant mass companion) the source radial velocity
jitter represents the minimum uncertainty with which \Gaia\ can measure 
radial velocity for a star of that apparent brightness and colour. However for 
stars with stellar companions and no evidence of double lines or binarity in the
cross-correlation-function, so-called SB1 type systems \citep{someone}, the
radial velocity jitter is a function of the orbital parameters and observation
epochs. We assume that the distribution of radial velocity jitter in the second
\Gaia\ data release is a mixture of two populations:

\begin{enumerate}
\item \emph{single-star systems}, where the radial velocity jitter represents the
      minimum radial velocity noise that \Gaia\ can measure for a star of that
      apparent magnitude and colour; and

\item \emph{higher-order star systems}, where the radial velocity jitter is
      significantly above that for single-star systems of a similar apparent
      magnitude and colour.
\end{enumerate}

We make a number of other explicit assumptions about the radial velocity
measurements reported by \Gaia:

\begin{itemize}
	\item the radial velocity jitter is a function of the apparent brightness 
		  of the point source
	\item the radial velocity jitter is a function of the apparent $bp-rp$
		  colour of the point source
	\item there is some intrinsic spread to the radial velocity jitter of
		  single stars, and that intrinsic spread also increases with the 
		  apparent brightness and colour of the source
	\item the difference between the true surface gravity $\logg$ and the
		  surface gravity of the radial velocity template used
		  (column \texttt{rv\_template\_logg}) has no impact on the radial
		  velocity jitter
	\item the difference between the true stellar metallicity [Fe/H] and
	 	  the metallicity of the radial velocity template used
		  (column \texttt{rv\_template\_feh}) has no impact on the radial
		  velocity jitter
	\item \todo{state something about teff template}		  
\end{itemize}

Conditioned on these assumptions, we adopt a two-component mixture model for 
the radial velocity jitter: a normal distribution where the mean and variance
are a function of stellar properties, and a uniform background component to
account for stellar multiples. Specifically, the likelihood is



\begin{eqnarray}
%	\mathcal{L} & \equiv & p\left(\{y
    MODEL
\end{eqnarray}

\noindent{}where \todo{X is Y} abd .

In principle for well-sampled circular orbits this model would allow us to
identify binary star systems with orbital periods up to twice the observing
span in the second \Gaia\ data release ($\approx$21 months observing span; 
$\approx$44 month orbital periods). For higher-order star systems with long 
periods, or with mass ratios that result in a small velocity semi-amplitude $K$, 
under our assumptions these systems may be classified as single-star systems 
because the intrinsic radial velocity variation is within the expected 
variations for single-stars of a similar apparent magnitude and colour. We 
revisit this issue in Section \ref{sec:sb_limits}, where we explicitly define 
the probability that a \Gaia\ source is a higher-order star system $p(y|\textrm{SB1})$ 
within defined limits of orbital period (and other orbital parameters). Outside of this
parameter range, we are insensitive to distinguishing single-star systems from
higher-order SB1-type systems.

\begin{itemize}
    \item show jitter compared to RP flux, etc
    \item how the model was fit
    \item show results of the fit
    \item what things did we \emph{not} include in the model (eg difference between temperature and the temperature of the template used), and why
    \item results
\end{itemize}



\subsection{SB2 Systems: Double-lined spectroscopic binaries}
\label{sec:sb2_methods}

Many sources in the second \Gaia\ data release do not have any reported radial 
velocity measurement, despite being bright enough ($G \lesssim 13$) and in a
suitable temperature range (e.g., between $X$ and $Y$) for radial velocities to 
be measured \citep{someone}. The principle reason why radial velocity measurements 
are \emph{not} reported for these stars is because the \Gaia/DPAC team have 
identified the source to be a double-lined spectroscopic binary (a SB2-type system), 
either through a composition of two stellar sources present in the spectra, or
from multiple (significant) modes in the cross-correlation function. In these
situations it is not sensible to report a point estimate of the radial velocity
of the point source. For fainter or bluer sources, however, the radial velocity 
may not be reported simply because it is too faint, blue, or red. 


We calculated the completeness of radial velocity measurements (e.g., the
fraction of sources with reported radial velocities) as a function of all 
available source properties that might affect whether the radial velocity may
be reported or not. This included position ($\alpha$, $\delta$, $l$, $b$),
parallax, proper motions, apparent magnitudes, colours, properties of the radial
velocity templates, stellar parameters ($\teff$, $\radius$, $\luminosity$),
and other properties. The full list of properties investigated is described in
Appendix \ref{appendix:sb2}. In Figure \ref{fig:radial_velocity_completeness} 
we show the completeness as a function of some pertinent properties, which demonstrate 
that the radial velocity completeness is relatively flat until a star becomes 
too faint (low \rp\ flux), or is either too blue or too red. We adopt 
conservative limits for when the radial velocity completeness starts to drop 
with these properties, and we assume that any source within the following
parameter range

\begin{eqnarray}
    \mathrm{phot\,\,rp\,\,mean\,\,flux} & \in & [10^{4}, 10^8] \nonumber \\
    \mathrm{bp\,\,-rp} & \in & [X, Y] 
    \label{eq:sb2_criteria}
\end{eqnarray}
\noindent{}is likely to be a double-lined spectroscopic binary (SB2) if no 
radial velocity is reported. That is to say that we are explicitly assuming
that if a point source meets the criteria in Equation \ref{eq:sb2_criteria} 
and does not have a radial velocity measurement, then the point source is
an unresolved double-lined spectroscopic binary. In principle we could
make more realistic attempts to model the radial velocity completeness as
a function of stellar properties rather than simply stating ``sources within
this parameter range should have radial velocities unless they are SB2 systems'',
but the radial velocity completeness within our specified range is reasonably
flat.


In Section \ref{sec:discussion} we detail some of the immediate limitations or 
interpretations of this assumption, but here we  note that the probability of a 
source being a double-lined spectroscopic binary is calculated (or rather, 
deduced) separately to our calculations of binarity probability given 
\emph{measured} quantities like radial velocity, astrometry, and photometric 
colours. And as we discuss in Section \ref{sec:discussion}, the main conclusions
of this work do not depend on SB2 systems. Nevertheless, we \emph{strongly} 
caution that our deductive inference of SB2 systems will be far more uncertain 
than the binary probabilities we derive from other information. Our indicator 
whether a star is likely a SB2 system likely suffers from more contamination
and lower completeness compared to the probabilities we derive from other information
(e.g., the radial velocity jitter, astrometric excess noise, and photometric 
colours), and the contamination and completeness likely vary over some
complex (unknown) function of parameter space. 



% Figure: radial velocity completeness as a function of pertinent properties
% 		  with the range that we have said is "fully complete"

% Figure: h-r diagram of SB2 binary fraction

% Figure: cross-match against a catalog of SB2 systems,... do we find all of
% 		  what they find from SB2 deductive inference alone? or do we miss
% 		  some SB2s in some parameter range?

% Figure: radial velocity fraction as a function of galactic latitude and
% 		  longitude


\subsection{Unresolved equal mass binaries: photometry}



\subsection{Astrometric binaries}

Many higher-order star systems are unresolved point sources in \Gaia, but the
stellar companion(s) have sufficient mass to measurably affect the tangential 
velocity of the primary and produce a jitter in astrometric positions. There are
many other sources of astrometric jitter, but here we show that the principle
effect of astrometric excess noise is due to stellar companions. Note that due
to the astrometry `degree of freedom bug' in the processing of the second \Gaia\
data release \citep[e.g., see Appendix A1 of ][]{Lindegren:2018}, we only make
use of astrometric excess noise from the \todo{Tycho-Gaia Astrometric Solution}
\citep{Michalik, TGAS, etc}. 



\begin{itemize}
    \item when would we expect AEN? what stellar or orbital properties does it 
          correlate or grow with?
    \item calibration model for AEN in DR1
\end{itemize}




\begin{figure*}
	\includegraphics[width=1.0\textwidth]{../figures/sb9-comparison.pdf}
    \caption{Source radial velocity variance shown with respect to flux in
    the \Gaia\ BP, G, and RP bands. The source radial velocity variance 
    most closely follows the RP flux as it is closer to the \Gaia\ RVS
    band.}
\end{figure*}

\begin{figure}
	\includegraphics[width=0.5\textwidth]{../figures/sb9-comparison.pdf}
    \caption{Source mean \Gaia\ RP flux and radial velocity variance.
    Points are represented by their mean $BP-RP$ color, confirming that
    there is some color dependency on the source radial velocity variance.}
    \label{fig:rp-flux-color}
\end{figure}


\begin{figure}
	\includegraphics[width=0.5\textwidth]{../figures/sb9-comparison.pdf}
    \caption{Mixture model fitted.}
    \label{fig:mixture-model}
\end{figure}



\begin{figure}
	\includegraphics[width=0.5\textwidth]{../figures/sb9-comparison.pdf}
    \caption{Detected binary fraction as a function of the mean RP flux
    (top panels), the $BP-RP$ color (second panels), the number of 
    radial velocity transits (third panels), and the inferred temperature
    (bottom panels). No trend is observed with mean RP flux or the number 
    of radial velocity transits. However, our model finds bluer (hotter)
    stars are more likely to be binaries, despite including color as a
    dependency in our model.}
    \label{fig:binary-fraction-wrt-params}
\end{figure}





\begin{itemize}

\item CALIBRATION SAMPLE (including ADQL query)

\item Fitting the model:

\item Comparisons of residuals with other stellar properties.

\item What is the RV error that Gaia can measure in a single epoch for a given magnitude vs colour, etc. 

\item What are the distributions of orbital parameters of binary systems that we would be able to detect?

\item Calculating a probability of binarity for every star, taking the non-measurements into account!

\item Completeness cuts,.... selection effects?

\end{itemize}

\section{Results}


\begin{itemize}

\item Binary fraction in the spaces that we were fitting: rp flux, colour, etc, just showing the transition of  probabilities

\item Binary fraction as a function of fitting properties, and other stellar properties from Gaia

\item Binarity across the H-R diagram

\item Binarity with metallicity 
\end{itemize}

\section{Discussion}

\subsection{Comparison with \citet{El_Badry:2017,El_Badry:2018}}

\subsection{Comparison with \citet{Badenes:2017}}

\subsection{Comparison with \citet{Raghavan:2010}}

\subsection{Comparison with \citet{Troup:2010}}

\subsection{Comparison with \citet{Price-Whelan:2010}}

\begin{itemize}
    \item \todo{binarity with metallicity}
    \item \todo{binarity across the H-R diagram}
    \item \todo{binarity in clusters vs the field?}
    \item \todo{binarity among extremely metal-poor stars?}
\end{itemize}


\acknowledgements

It is a pleasure to thank
    Adrian Price-Whelan,
	David W. Hogg,
	Lorenzo Spina, 
		and 
	Kevin C. Schlaufman.
This work was partly supported through the Australian Research Council 
through Discovery Grant DP160100637.
This work has made use of data from the European Space Agency (ESA) mission {\it
Gaia} (\url{https://www.cosmos.esa.int/gaia}), processed by the {\it Gaia} Data
Processing and Analysis Consortium (DPAC,
\url{https://www.cosmos.esa.int/web/gaia/dpac/consortium}). Funding for the DPAC
has been provided by national institutions, in particular the institutions
participating in the {\it Gaia} Multilateral Agreement.  This research was
started at the NYC Gaia DR2 Workshop at the Center for Computational
Astrophysics of the Flatiron Institute in 2018 April.

This work has made use of CosmoHub. CosmoHub has been developed by the Port 
d'Informaci\'o Cient\'ifica (PIC), maintained through a collaboration of the 
Institut de F\'isica d'Altes Energies (IFAE) and the Centro de Investigaciones 
Energ\'eticas, Medioambientales y Tecnol\'ogicas (CIEMAT), and was partially 
funded by the ``Plan Estatal de Investigaci\'on Cient�fica y T\'ecnica y de 
Innovaci\'on'' program of the Spanish government.




\software{
	\package{Astropy} \citep{astropy},
    \package{IPython} \citep{ipython},
    \package{matplotlib} \citep{mpl},
    \package{numpy} \citep{numpy},
    \package{scipy} \citep{scipy},
    \package{Stan} \citep{stan},
    \package{CosmoHub} \citep{cosmohub}
    %tensorflow
}    

\bibliographystyle{aasjournal}
\bibliography{velociraptor}



\end{document}
