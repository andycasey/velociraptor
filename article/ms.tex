\documentclass[twocolumn]{aastex61}
\usepackage{bm}

\newcommand\teff{T_{\rm eff}}
\newcommand\logg{\log{g}}
\newcommand\feh{[\rm{Fe}/\rm{H}]}
\newcommand\mh{[\rm{M}/\rm{H}]}

\newcommand{\project}[1]{\textsl{#1}}
\newcommand{\acronym}[1]{{\small{#1}}}
\newcommand{\Gaia}{\project{Gaia}}
\newcommand{\gaia}{\project{gaia}}

\received{2018 XX XX}
\revised{2018 XX XX}
\accepted{2018 XX XX}

%\submitjournal{AAS Journals}

\shorttitle{Short title}
\shortauthors{Casey et al.}

\begin{document}


\title{Binarity and other things in Gaia}

\correspondingauthor{Andrew R. Casey}
\email{andrew.casey@monash.edu}

\author[0000-0003-0174-0564]{Andrew R. Casey}
\affiliation{School of Physics \& Astronomy, Monash University, Clayton 3800, 
    Victoria, Australia}
\affiliation{Faculty of Information Technology, Monash University, Clayton 3800, 
    Victoria, Australia}

\author[0000-0003-2866-9403]{David W. Hogg}
\affiliation{Center for Cosmology and Particle Physics, Department of Physics,
    New York University, 726 Broadway, New York, NY 10003, USA}
\affiliation{Center for Data Science, New York University, 60 Fifth Ave, 
    New York, NY 10011, USA}
\affiliation{Max-Planck-Institut f\"ur Astronomie, K\"onigstuhl 17, D-69117 
    Heidelberg}
\affiliation{Flatiron Institute, 162 Fifth Ave, New York, NY 10010, USA}

\begin{abstract}
Velociraptors can't see you if you don't move. (\Gaia can see you if you can.)
\end{abstract}


\keywords{stars: dynamics}

\section{Introduction} \label{sec:intro}

\begin{itemize}
\item the binary fraciton is important
\item Gaia provides exquisite astrometry and RV measurements over many years
\item The errors in these quantities are well-characterised
\item When a star is in a binary system (or triary, etc), there will be excess scatter relative to a single star of the primary's properties
\item the excess RV noise (and in DR1 or DR3, excess astrometric noise) can be used to infer the presence of a companion
\end{itemize}


\section{Model}

% What is provided by Gaia?
% astrometric excess, number of measurements, brightness, etc
% what is the error model that Gaia uses?
% WE COULD USE A HIErARCHICAL MODEL FOR THE ERRORS
% Thus for a single star we would expect, so we can calculate
% p(\ra, \dec|data)


% For stars with a companion, if the companion is large enough to induce an effect on the primary then this would produce astrometric noise that would  be observable in the


% Astrometric model:


% Can we do this for DR1 already?

% When would we expect to see astrometric noise and not RV noise, and vice-versa? in what situations?


% we can calculate probability that it is consistent with a single  star,... p-value?


% From the Gaia expected error distributions: how many stars might we be able to discriminate between?

\section{Results}


\section{Discussion}



\software{numpy, scipy, astropy, Stan}

\acknowledgements
- Gaia/DPAC
- Gaia archive?
- Gaia DR2 hack week at Flatiron


\end{document}
