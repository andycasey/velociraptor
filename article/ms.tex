\documentclass[twocolumn]{aastex61}
\usepackage{bm}
\usepackage{color}

\newcommand\teff{T_{\rm eff}}
\newcommand\logg{\log{g}}
\newcommand\feh{[\rm{Fe}/\rm{H}]}
\newcommand\mh{[\rm{M}/\rm{H}]}

\newcommand{\project}[1]{\textsl{#1}}
\newcommand{\package}[1]{\texttt{#1}}
\newcommand{\acronym}[1]{{\small{#1}}}
\newcommand{\Gaia}{\project{Gaia}}
\newcommand{\gaia}{\project{gaia}}
\newcommand{\todo}[1]{\textcolor{red}{#1}}


\newcommand{\GaiaRVE}{\sigma_{\mathrm{V}_\mathrm{R}}^\mathrm{MTA}}
\newcommand{\RVJitter}{\sigma(\mathrm{V}_\mathrm{R}^{t})}

\received{2018 XX XX}
\revised{2018 XX XX}
\accepted{2018 XX XX}

%\graphicspath{../figures}

%\submitjournal{AAS Journals}

\shorttitle{Stellar multiplicity}
\shortauthors{Casey et al.?}

\begin{document}

\title{Stellar multiplicity in \Gaia}

\correspondingauthor{Andrew R. Casey}
\email{andrew.casey@monash.edu}

\author[0000-0003-0174-0564]{Andrew R. Casey}
\affiliation{School of Physics \& Astronomy, Monash University, Clayton 3800, 
    Victoria, Australia}
\affiliation{Faculty of Information Technology, Monash University, Clayton 3800, 
    Victoria, Australia}

%\author{Daniel Foreman-Mackey}
%\affiliation{Flatiron Institute, 162 Fifth Ave, New York, NY 10010, USA}


%\author[0000-0003-2866-9403]{David W. Hogg??}
%\affiliation{Center for Cosmology and Particle Physics, Department of Physics,
%    New York University, 726 Broadway, New York, NY 10003, USA}
%\affiliation{Center for Data Science, New York University, 60 Fifth Ave, 
%    New York, NY 10011, USA}
%\affiliation{Max-Planck-Institut f\"ur Astronomie, K\"onigstuhl 17, D-69117 
%    Heidelberg}
%\affiliation{Flatiron Institute, 162 Fifth Ave, New York, NY 10010, USA}

\begin{abstract}
Most stars are in binary systems. 
The occurrence rate of binary star systems is critically important to many 
fields of astrophysics, but the effects of binaries are commonly ignored in 
astronomical data analysis applications.
Here we construct a model to probabilistically identify binary stars based on
excess radial velocity scatter, photometry, and astrometric excess noise.

We identify binaries \todo{within some orbital and SP limits} and show that 
the orbital semi-amplitude is well-approximated by the excess radial velocity
scatter for systems within these criteria.

\todo{We make a few interesting tests about binarity...}
\end{abstract}


\keywords{stars: dynamics}

\section{Introduction} \label{sec:intro}

\begin{itemize}
\item the binary fraciton is important
\item Gaia provides exquisite astrometry and RV measurements over many years
\item The errors in these quantities are well-characterised
\item When a star is in a binary system (or triary, etc), there will be excess scatter relative to a single star of the primary's properties
\item the excess RV noise (and in DR1 or DR3, excess astrometric noise) can be used to infer the presence of a companion
\end{itemize}


\section{Method}

There are at least three ways that binary systems can be reliably
identified using from \Gaia data: through radial velocity variations, 
anomalous photometric colours, or excess astrometric noise. In this 
work we do not make use of astrometric noise because of the 
degree-of-freedom astrometry bug present in the second \Gaia\ data 
release \citep{Lindegren:2018}. We will describe the radial velocity
information that forms part of our model for identifying binary stars,
and then discuss the photometric information used, before explicitly
outlining the assumptions of our model.


The second \Gaia\ data release provides median radial velocity 
estimates for 7,224,631 stars. The radial velocity error provided
($\GaiaRVE$; \texttt{radial\_velocity\_error}) is a function of the 
number of transits $N$ (i.e., observations; column 
\texttt{rv\_nb\_transits}) and the scatter among those $N$ measurements. 
This allows us to calculate the scatter among those measurements, or
rather the \emph{radial velocity jitter}, for each source
\begin{eqnarray}
\RVJitter = \sqrt{\frac{2N}{\pi}}\GaiaRVE \quad .
\end{eqnarray}



For a single star system observed $N$ times, the source radial velocity
jitter represents either the minimum uncertainty with which
\Gaia\ can measure radial velocity for a star of that apparent 
brightness and colour. However for stars with stellar companions,
the radial velocity scatter is a function of orbital parameters 
and observing epochs. Thus we can reasonably assume that the
radial velocity jitter will be a mixture of
two populations: single stars, where the radial velocity jitter
represents the minimum radial velocity noise that \Gaia\ can
measure for a star of that type, and a binary population that
could have radial velocity variances well above the noise floor.



\begin{figure*}
	\includegraphics[width=1.0\textwidth]{../figures/sb9-comparison.pdf}
    \caption{Source radial velocity variance shown with respect to flux in
    the \Gaia\ BP, G, and RP bands. The source radial velocity variance 
    most closely follows the RP flux as it is closer to the \Gaia\ RVS
    band.}
\end{figure*}

\begin{figure}
	\includegraphics[width=0.5\textwidth]{../figures/sb9-comparison.pdf}
    \caption{Source mean \Gaia\ RP flux and radial velocity variance.
    Points are represented by their mean $BP-RP$ color, confirming that
    there is some color dependency on the source radial velocity variance.}
    \label{fig:rp-flux-color}
\end{figure}


\begin{figure}
	\includegraphics[width=0.5\textwidth]{../figures/sb9-comparison.pdf}
    \caption{Mixture model fitted.}
    \label{fig:mixture-model}
\end{figure}



\begin{figure}
	\includegraphics[width=0.5\textwidth]{../figures/sb9-comparison.pdf}
    \caption{Detected binary fraction as a function of the mean RP flux
    (top panels), the $BP-RP$ color (second panels), the number of 
    radial velocity transits (third panels), and the inferred temperature
    (bottom panels). No trend is observed with mean RP flux or the number 
    of radial velocity transits. However, our model finds bluer (hotter)
    stars are more likely to be binaries, despite including color as a
    dependency in our model.}
    \label{fig:binary-fraction-wrt-params}
\end{figure}





% we assume what?
%

% given these assumptions, the model we adopt is XYZ
% we chose to model in RP flux as it more closely represents the flux counts that would be received by the Gaia RVS, rather than G or BP flux.

% the model we use is

%  

\begin{itemize}
\item The effect of binaries.

\item RV noise vs astrometric excess noise.

\item When would we expect to see astrometric noise and not RV noise, and vice-versa? in what situations?

\item Here we base our model using RV data only from Gaia DR2.

\item Some fraction of stars in Gaia are single star systems (or have such low semi-amplitudes that would not be detected by Gaia), and we will use those stars (within a mixture model) to calibrate the radial velocity error floor

\item We will do this on individual epoch measurements

\item ASSUMPTIONS

\item MODEL

\item CALIBRATION SAMPLE (including ADQL query)

\item Fitting the model:

\item Comparisons of residuals with other stellar properties.

\item What is the RV error that Gaia can measure in a single epoch for a given magnitude vs colour, etc. 

\item What are the distributions of orbital parameters of binary systems that we would be able to detect?

\item Calculating a probability of binarity for every star, taking the non-measurements into account!

\item Completeness cuts,.... selection effects?

\end{itemize}

\section{Results}

\begin{itemize}

\item Binary fraction in the spaces that we were fitting: rp flux, colour, etc, just showing the transition of  probabilities

\item Binary fraction as a function of fitting properties, and other stellar properties from Gaia

\item Binarity across the H-R diagram

\item Binarity with metallicity 

\item Binarity at height above the plane?

\item Binarity in clusters vs the plane?
\end{itemize}

\section{Discussion}


\acknowledgements

It is a pleasure to thank
	David W. Hogg,
	Lorenzo Spina, 
		and 
	Kevin C. Schlaufman.
This work was partly supported through the Australian Research Council 
through Discovery Grant DP160100637.
This work has made use of data from the European Space Agency (ESA) mission {\it
Gaia} (\url{https://www.cosmos.esa.int/gaia}), processed by the {\it Gaia} Data
Processing and Analysis Consortium (DPAC,
\url{https://www.cosmos.esa.int/web/gaia/dpac/consortium}). Funding for the DPAC
has been provided by national institutions, in particular the institutions
participating in the {\it Gaia} Multilateral Agreement.  This research was
started at the NYC Gaia DR2 Workshop at the Center for Computational
Astrophysics of the Flatiron Institute in 2018 April.

This work has made use of CosmoHub. CosmoHub has been developed by the Port d'Informaci\'o Cient\'ifica (PIC), maintained through a collaboration of the Institut de F\'isica d'Altes Energies (IFAE) and the Centro de Investigaciones Energ\'eticas, Medioambientales y Tecnol\'ogicas (CIEMAT), and was partially funded by the "Plan Estatal de Investigaci\'on Cient�fica y T\'ecnica y de Innovaci\'on" program of the Spanish government.




\software{
	\package{Astropy} \citep{astropy},
    \package{IPython} \citep{ipython},
    \package{matplotlib} \citep{mpl},
    \package{numpy} \citep{numpy},
    \package{scipy} \citep{scipy},
    \package{Stan} \citep{stan},
    \package{CosmoHub} \citep{cosmohub}
    %tensorflow
}    

\bibliographystyle{aasjournal}
\bibliography{velociraptor}



\end{document}
