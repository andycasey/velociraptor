\documentclass[twocolumn]{aastex61}
\usepackage{bm}
\usepackage{amsmath}
\usepackage{color}
\usepackage{comment}
\usepackage{minted}

\newcommand\teff{T_{\rm eff}}
\newcommand\logg{\log{g}}
\newcommand\feh{[\rm{Fe}/\rm{H}]}
\newcommand\mh{[\rm{M}/\rm{H}]}
\newcommand{\luminosity}{L_\circ}
\newcommand{\radius}{R_\circ}


\newcommand{\project}[1]{\textsl{#1}}
\newcommand{\package}[1]{\texttt{#1}}
\newcommand{\acronym}[1]{{\small{#1}}}
\newcommand{\ESA}{\acronym{ESA}}
\newcommand{\Gaia}{\project{Gaia}}
\newcommand{\gaia}{\project{gaia}}
\newcommand{\todo}[1]{\textcolor{red}{#1}}
\newcommand{\rp}{\textsl{rp}}
\newcommand{\bp}{\textsl{bp}}


\newcommand{\NumberOfStellarMultiples}{XXX}
\newcommand{\NumberOfStellarSingles}{XXX}
\newcommand{\NumberOfSourcesInSubset}{8,911,018} % check

\newcommand{\GaiaRVE}{\sigma_{\mathrm{V}_\mathrm{R}}^\mathrm{MTA}}
\newcommand{\RVJitter}{\sigma(\mathrm{V}_\mathrm{R}^{t})}

\received{2018 XX XX}
\revised{2018 XX XX}
\accepted{2018 XX XX}

%%% This file is generated by the Makefile.

\newcommand{\vcpath}{vc.tex}

\IfFileExists{\vcpath}{\input{\vcpath}}{
	\newcommand{\giturl}{UNKNOWN}
	\newcommand{\gitslug}{UNKNOWN}
	\newcommand{\githash}{UNKNOWN}
	\newcommand{\gitdate}{UNKNOWN}
	\newcommand{\gitauthor}{UNKNOWN}
}


\submitjournal{AAS Journals}

\shorttitle{Stellar multiplicity}
\shortauthors{Casey et al.}

\begin{document}

\title{Detection and partial characterisation of stellar multiplicity with Gaia}

\correspondingauthor{Andrew R. Casey}
\email{andrew.casey@monash.edu}

% Some order of the following authors (order is TBD)

\author[0000-0003-0174-0564]{Andrew R. Casey}
\affiliation{School of Physics \& Astronomy, 
			 Monash University,
			 Wellington Rd, Clayton 3800, Victoria, Australia}
\affiliation{Faculty of Information Technology, 
			 Monash University, 
			 Wellington Rd, Clayton 3800, Victoria, Australia}

\author[0000-0002-9328-5652]{Daniel Foreman-Mackey}
\affiliation{Flatiron Institute, 
			 162 Fifth Ave, New York, NY 10010, USA}


\author[0000-0003-3494-343X]{Carles Badenes}
\affiliation{Department of Physics and Astronomy, 
			 University of Pittsburgh, 
			 3941 O'Hara Street, Pittsburgh, PA 15260, USA}

\author{Rosemary Mardling}
\author{John Lattanzio}

\author[0000-0003-2866-9403]{David W. Hogg}
\affiliation{Center for Cosmology and Particle Physics, Department of Physics,
    		 New York University, 
		  	 726 Broadway, New York, NY 10003, USA}
\affiliation{Center for Data Science, 
			 New York University, 60 Fifth Ave, New York, NY 10011, USA}
\affiliation{Max-Planck-Institut f\"ur Astronomie, 
			 K\"onigstuhl 17, D-69117 Heidelberg}
\affiliation{Flatiron Institute,
			 162 Fifth Ave, New York, NY 10010, USA}

\author[0000-0003-0872-7098]{Adrian M. Price-Whelan}
\affiliation{Department of Astrophysical Sciences, 
			 Princeton University, 
			 Princeton, NJ 08544, USA}




\begin{abstract}
The frequency and properties of stellar multiples (e.g., binaries, triples) underpin 
much of astrophysics, and are difficult to estimate.  
Here we use \Gaia\ data to detect and partially characterise \NumberOfStellarMultiples\
stellar multiple systems based on excess jitter in astrometry and radial velocity.
We reliably detect binary systems with orbital periods up to $\approx$3.5\,yr. % and systems with what kind of mass ratios, semi-amplitudes,  etc. 
We use eclipsing binary systems to calibrate excess astrometric jitter with
tangential motion, allowing us to estimate inclination angles and directly constrain
companion masses for millions of systems.
\todo{We find that the distribution of inclination angles is isotropic.}
\todo{A comment on stellar multiplicity with stellar metallicity.}
\todo{A comment on stellar multiplicity in the field relative to clusters.}
\todo{A comment on systems with stellar remnants (e.g., NS/BHs).}
\end{abstract}


\keywords{(stars:) binaries: general, (stars:) binaries: spectroscopic, (stars:) binaries (including multiple): close, astrometry, techniques: radial velocities, methods: statistical}

\section{Introduction} \label{sec:intro}

A higher order stellar system describes any star that has at least one stellar 
companion (e.g., binaries, trinaries). The presence of stellar companions
complicates inferences on stellar and galactic properties, but these factors are
often ignored because it is extremely challenging to detect stellar multiplicity
for an unresolved system, and even harder to separate the contributions from individual
sources.

The \Gaia\ space telescope provides exquisite astrometry, photometry, and radial
velocity measurements over many years for million of point sources in our galaxy
\citep{GaiaDR2}.
When a point source has an unresolved stellar companion within some range of orbital
and stellar parameters, it is expected that there will be an excess in motion jitter
relative to a single star of a similar type. This is nothing new: astronomers have 
used radial velocity \citep{Butler:1996} and astrometric variations \citep{Muterspaugh:2010}
to infer the presence of stellar and exoplanet companions for decades. However, 
the \Gaia\ data presents an opportunity to infer the presence of stellar companions
for millions of point sources across both hemispheres. No ground- or space-based 
instrument has ever provided such data.

\todo{Literature review on what we do and don't know about  the multiplicity fraction}

In this work we make use of astrometry (including radial motion) \todo{and photometry?}
from the second \Gaia\ data release to infer the presence of stellar companions 
for millions of point sources. We provide partial characterisations of these systems,
based on the data available for each source. In Section \ref{sec:method} we 
describe our methods, and in Section \ref{sec:results} we outline our results 
in context of other stellar multiplicity surveys. We include comparisons of 
binary properties between clusters and the field, as a function of stellar 
properties (e.g., stellar metallicity), and highlight some particularly 
noteworthy systems that would benefit from additional observations. 


\section{Data} \label{sec:data}

There are many ways that binaries can be reliably identified using \Gaia\
data, including radial velocity measurements, astrometry, and photometry.
These data are sensitive to characterising stellar multiplicity in 
different ways. Here we describe the data that we use from the second
\Gaia\ data release \citep{Lindegren:2018} before defining our model.

%Here we define a model that we will use to detect and 
%characterise stellar multiples from radial velocity and astrometric data,
%before describing how the absence of \Gaia\ radial velocities can also be
%used to infer stellar multiplicity.

%Here we describe the absence of reported radial velocity in
%\Gaia\ data can be used to deduce stellar multiples, before we describe
%our model that incorporates measurements of radial velocity, astrometry, 
%and photometry, to detect and characterise stellar multiples.

\subsection{Radial velocities}
\label{sec:data_vrad}

Radial velocity measurements are not available for individual epochs in the 
second \Gaia\ data release \citep{Lindegren:2018,Cropper:2018}, but a point estimate
and an error in radial velocity is available for 7,224,631 sources \citep{Cropper:2018}.
% TODO: That's the value from the paper, but I think the true value is lower. Check.
The reported radial velocity error ($\GaiaRVE$; column name \texttt{radial\_velocity\_error}) 
is a function of the number of transits $N$ (i.e., the number of observations; 
given by column \texttt{rv\_nb\_transits}) and the standard deviation among 
those measurements. This allows us to calculate the standard deviation in radial 
velocity for each source, independent of the number of measurements,
which we will refer to as the \emph{radial velocity jitter} $\RVJitter$,
\begin{eqnarray}
\RVJitter = \sqrt{\frac{2N}{\pi}}\GaiaRVE \quad .
\end{eqnarray}



For a single star, or a star without a significant mass companion, the source radial
velocity jitter represents the minimum uncertainty with which \Gaia\ can measure 
radial velocity for a star of that colour, apparent magnitude, and absolute magnitude.
Stars with bluer colours have fewer absorption lines with deeper wings, and fainter stars
have on average lower signal-to-noise (S/N) ratios, both of which result in noisier radial
velocity measurements. The absolute magnitude is similarly important, as giant stars have
narrow absorption lines than main-sequence stars of the same temperature and colour. For stellar multiples, the source radial velocity jitter will be the quadrature sum of the expected jitter for the primary (if it were a single star) and a contribution from the radial velocity semi-amplitude of the orbit.

Some sources in \Gaia\ do not have reported radial velocities even though they are
apparently bright enough and within a suitable \bp\ - \rp\ colour range.  The likely
explanation for these situations is that the \Gaia\  team identified
the source to be a double-lined spectroscopic binary, and there is no utility in
reporting a point estimate of the \emph{source} radial velocity and associated error.
Other explanations are also possible, which implies that we cannot use the \emph{lack}
of radial velocity as a robust identifier of stellar multiplicity. \todo{In the following
Sections}
% Show it ehre?
we demonstrate that for sources that do not have reported radial velocities -- and 
sources of similar apparent magnitude and colour \emph{do} have radial velocities --
their astrometric jitter is usually sufficient to distinguish single stars from stellar
multiples.

% todo:  Figure showing distribution of astrometric unit weight error.


\subsection{Astrometry} \label{sec:data_astrometry}

Many higher-order star systems are unresolved point sources in \Gaia, but the
stellar companion(s) may have sufficient mass to measurably affect the tangential 
velocity of the primary and produce a detectable jitter in the astrometric 
position. 
%There are many other sources of astrometric jitter, but here we show 
%that the principle effect of astrometric excess noise is due to stellar 
%companions.
We make use of the astrometric unit weight error $u$ to measure astrometric
jitter
\begin{eqnarray}
	u = \sqrt{\frac{\chi^2_{al}}{N_{obs,al} - 5}} \label{eq:astrometric_unit_weight_error}
\end{eqnarray}
\noindent{}where $\chi^2_{al}$ is the astrometric $\chi^2$ value
in the along-scan direction (column \texttt{astrometric\_chi2\_al}) and 
$N_{obs,al}$ is the number of astrometric observations in the along-scan
direction (column \texttt{astrometric\_n\_obs\_al}). 
The astrometric unit weight error is unaffected by the `degree of freedom bug'
that occurred during the processing of the the second \Gaia\ data release
\citep[e.g., see Appendix A1 of ][]{Lindegren:2018}, which resulted in 80\% of
sources having zero astrometric excess noise. 
The astrometric unit weight error can be interpreted similarly to how the
astrometric excess noise would be interpreted (e.g., large values are less
consistent with a single star model), but using the astrometric unit weight
error allows us to distinguish astrometric binaries in a much larger sample.
And like the radial velocity jitter, the unit weight astrometric error of a
single source must be considered in context with stars of similar colour 
and apparent magnitude (but not necessarily absolute magnitude).


% todo: what ADQL queries did we do?
% what will our subset be?
\subsection{\texttt{ADQL} query}

We restrict our work to the brightest $\sim{}10^7$ sources in \Gaia\ DR2.
About 70\% of sources in this subset have reported radial velocities,
allowing us to identify spectroscopic binaries, and this subset provides
a good sample of sources to test the precision to which we can identify
astrometric binaries. Specifically, we executed the following Astronomical Data Query
Language \citep[\texttt{ADQL};][]{someone} query on \ESA\ \Gaia\ archive
\citep{esa_gaia_archive}:

\begin{minted}[style=friendly]{postgresql}
SELECT *
  FROM gaiadr2.gaia_source
 WHERE radial_velocity IS NOT NULL
    OR phot_rp_mean_mag <= 13
\end{minted}
\noindent{}This query returned \NumberOfSourcesInSubset\ sources.


\section{Method} \label{sec:method}

We assume that for a sample of sources with similar \bp\ - \rp\ colours
and apparent \rp\ magnitudes -- and absolute \rp\ magnitudes, if radial velocities
are being considered -- the distribution of jitter is a mixture of two components:

\begin{enumerate}
\item \emph{single-star systems}, where the jitter represents the minimum noise that 
	   \Gaia\ can measure for a source with those properties (e.g., a source with
	   that \bp\ - \rp\ colour, apparent \rp\ magnitude, and perhaps absolute \rp\
	   magnitude);
      
\item \emph{higher-order star systems}, where the jitter is significantly higher 
	  than what is measured for single-star systems of similar properties.
\end{enumerate}

We further assume that the jitter for a single star will change depending on the
\bp\ - \rp\ colour, the apparent magnitude, and the absolute magnitude. For example, 
the mean radial velocity jitter of single stars with $\teff \approx 6000\,\textrm{K}$
will be higher than the jitter of single stars with $\teff \approx 5000\,\textrm{K}$, 
with all else being (nearly) equal. Similarly, the minimum radial velocity jitter
for giant stars is expected to be lower on average than for main-sequence stars of
the same colour, as giant stars have narrower absorption lines than main-sequence
stars of the same colour, allowing for a more precise measurement of the radial
velocity. However, this effect will not affect the astrometric jitter: the minimum
astrometric jitter for a main-sequence star and a giant star of the same temperature
is likely to be indistinguishable. For these reasons, in order to evaluate whether 
a source is more likely to be a single star or a stellar multiple, we must consider 
the jitter in context with other sources that have a similar \bp\ - \rp\ colour, 
apparent \rp\ magnitude, and in the case of radial velocity jitter, the absolute
\rp\ magnitude.\footnote{The magnitude in any band would be sufficient. Throughout
this work we chose the \rp\ band because its wavelength range most closely overlaps 
with that of the Radial Velocity Spectrograph on board \Gaia.} However, we cannot reliably describe a quantitative
relationship for \emph{how} exactly the astrometric and radial velocity jitter is
expected to vary for a single star given some source properties. For these reasons
we chose to adopt a flexible non-parametric model for stellar multiplicity, primarily
conditioned on the assumption that if we select enough sources with \emph{similar}
properties then within that subset of similar stars there exists \emph{some} single stars 
and \emph{some} stellar multiples.

We illustrate our approach in Figure \ref{fig:npm_schematic}, which can be qualitatively
described by the following steps. When considering the radial velocity jitter for a
\Gaia\ source of interest, we select $N$ sources that have a similar \bp\ - \rp\ colour,
apparent \rp\ magnitude, and absolute \rp\ magnitude. We then fit the distribution of
radial velocity jitter $\RVJitter$ as a mixture of two components: a normal distribution
to represent single stars, and a log-normal distribution to represent stellar multiples.
We perform these steps for $\approx10^4$ sources. While this process approximately maps
how the radial velocity jitter of single (and multiple) stars varies across the Hertszprung-Russell
diagram, each source of interest is considered independent from all others. For example,
for three sources that are all indistinguishable from each other in \bp\ - \rp\ colour, 
apparent \rp\ magnitude, and absolute \rp\ magnitude, the procedure we described above
treats the fitting of these three mixture models as entirely independent of each other,
even though we suspect that the parameters of the three mixture models ought to be 
identical. In other words: the procedure as described does not enforce any smoothness.
Given enough data this is not a problem for densely-populated regions of the 
Hertszprung-Russell diagram as the optimized results \emph{will} be (statistically)
indistinguishable for similar sources. In practice this poses a problem for poorly-sampled
regions of the Hertzsprung-Russell diagram, and raises subjective questions on what
constitutes \emph{the most} similar type of source.

\todo{In the third step we apply a Gaussian process.}


\begin{figure*}
	\includegraphics[width=1.0\textwidth]{../figures/todo.png}
    \caption{Illustration of the method we adopt for identifying and characterising
    		 stellar multiplicity across the Hertsprung-Russell diagram. For each
		 	 \Gaia\ source we select between 128 and 1024 sources that have
			 similar \bp\ - \rp\ colours, apparent \rp\ magnitudes, and absolute \rp\
			 magnitudes (Step 1). With these similar sources we fit a two-component
			 model to distinguish single star sources from stellar multiples (Step 2).
			 With the model parameters optimised from Step 2, we evaluate probability
			 of stellar multiplicity and partially characterise the higher order
			 systems by assuming binarity and estimating the system semi-amplitude (Step 3).}
    \label{fig:npm_schematic}
\end{figure*}

% todo: joining sentence(s) here?

\subsection*{Step 1: Selecting similar sources}

We construct a $k$-dimensional tree \citep[k-d tree;][]{someone} for all sources in the
dimensions of \bp\ - \rp\ colour, apparent \rp\ magnitude, and absolute \rp\ magnitude.
When doing so we scale the dimensions such that \todo{0.1}\,mags in \bp\ - \rp\ colour is
approximately equal to \todo{1} magnitude in apparent \rp\ magnitude and \todo{1} magnitude in 
absolute \rp\ magnitude. We use this $k$-d tree for selecting a `ball' of similar
stars for a given source.

We enforce a number of constraints when selecting similar stars. For each source
we require the multi-dimensional `ball' to include at least \todo{128} sources.
We further require that the radius of the ball extends to at least \todo{0.05}\,mags in 
\bp\ - \rp\ colour, \todo{0.5}\,mags in apparent \rp\ magnitude, and \todo{0.5}\,mags in absolute
\rp\ magnitude. If the initial query of \todo{128} points does not extend wide enough to
meet our constraints on minimum radius, we extend the radius (linearly in each scaled)
dimension until the minimum constraints on radii are met. Our minimum radii constraints
imply that there will be many (e.g., $>>\todo{128}$) points returned in dense regions of the
Hertszprung-Russell diagram. In these situations we then subsample the selected points
to return a random \todo{1024} sources in any ball, while enforcing our constraints on
minimum radii. As a result, the number of `similar' sources selected will range from
a minimum of \todo{128} to a maximum of \todo{1024}, and the minimum radii constraints on various
dimensions are always maintained.

% maximum ball radius


\subsection*{Step 2: Fitting the two-component mixture}

We fit a two-component mixture model to the distribution of radial velocity jitter for
all similar sources selected through Step 1. This mixture model includes a normal
distribution to represent the intrinsic jitter for single stars, and a log-normal
distribution to represent stellar multiples. Specifically the model parameters include
a mixing weight parameter $\theta$ that is constrained between $(0, 1)$, the mean 
$\mu_{s}$ and standard deviation $\sigma_{s}$ of the normal distribution,
and the mean $\mu_{m}$ and standard deviation $\sigma_{m}$ of the 
log-normal distribution. 

We assume a beta prior on $\theta$  such that $\theta \sim \mathcal{B}\left(5, 5\right)$
and require that $\sigma_{s}$ and $\sigma_{m}$ are larger than zero.
Depending on the distribution of the radial velocity jitter, we often found this mixture
challenging to optimize because the log-normal distribution could easily account for 
most of the observed distribution, which tended the mixing weight to zero and implied
that all selected sources were stellar multiples. To avoid this situation we placed a
constraint on the mean of the log-normal distribution to ensure that the \emph{mode}
of the log-normal distribution is larger than $\mu_{s} + 1\sigma_{s}$ such that
\begin{eqnarray}
%	\mu_{s} + 5\sigma_{s} & > & \exp\left(\mu_{m} - \sigma_{m}^2\right) \label{eq:rvm_lower_bound} \\
	\mu_{s} + 1\sigma_{s} & < & \exp\left(\mu_{m} - \sigma_{m}^2\right) \label{eq:rvm_upper_bound} \quad .
\end{eqnarray}

\begin{comment}
However, we found that this constraint alone was insufficient. Often the mode would
be very close to the $\mu_{s} + 1\sigma_{s}$ lower bound, which implies that
the log-normal distribution is providing a considerable fraction of probability support
to sources with radial velocity jitter near the mean of the \emph{normal} distribution.
For this reason we chose to alter our lower bound constraint in that we also required
that at the mean of the normal distribution, the cumulative probability distribution
of the log-normal must be less than 10\% (let $q = 0.10$). The cumulative distribution 
function for the log-normal distribution is
\begin{eqnarray}
\textrm{CDF}(x|\mu_{m},\sigma_{m}) = \frac{1}{2} + \frac{1}{2}\textrm{erf}\left(\frac{\log{x} - \mu_{m}}{\sqrt{2}\sigma_{m}}\right) 
\end{eqnarray}

\noindent{}where \textrm{erf} is the standard error function. We use the approximation 
to the error function 
\begin{eqnarray}
	\textrm{erf}\left(x\right) \approx \tanh\left(\sqrt{\pi}\log{2}x\right)
\end{eqnarray}

\noindent{}and use the identity
\begin{eqnarray}
	\tanh^{-1}\left(x\right) = \frac{1}{2}\left[\log{\left(1 + x\right)} - \log{\left(1 - x\right)}\right]
\end{eqnarray}

\noindent{}to write our constraint on $\mu_{m}$ as
\begin{eqnarray}
	\mu_{m} & > & \log\mu_{s} + \frac{\sigma_{m}}{2\log{2}}\sqrt{\frac{2}{\pi}}\log{\left(\frac{1}{q} - 1\right)} \nonumber \\
	\mu_{m} & > & \log\mu_{s} + \frac{\sigma_{m}}{2\log{2}}\sqrt{\frac{2}{\pi}}\log{9} \label{eq:rvm_lower_bound2} \quad .
\end{eqnarray}

Incorporating the bounds from Equations \ref{eq:rvm_lower_bound}, \ref{eq:rvm_upper_bound},
and \ref{eq:rvm_lower_bound2} leads to the prior
\begin{eqnarray}
	\mu_{m} \sim \mathcal{U}\left(\alpha, \beta\right)
\end{eqnarray}

\noindent{}where:
\begin{eqnarray}
  \alpha & = & \max\left(\!
	\begin{aligned}
		\log{\left(\mu_{s} + 1\sigma_{s}\right)} + \sigma_{m}^2\\
		\log\mu_{s} + \frac{\sigma_{m}}{2\log{2}}\sqrt{\frac{2}{\pi}}\log{9}
	\end{aligned}\right) \\
  \beta & = & \log{\left(\mu_{s} + 5\sigma_{s}\right)} + \sigma_{m}^2 \quad .
\end{eqnarray}

We found these priors were sufficient to ensure numerical stability and efficiency.
\end{comment}

Wherever possible we initialised the model parameters from nearby points (within the
ball selected by the k-d tree) where an optimised result already exists. This was primarily to minimise
computational cost; the initialisation did not have an impact on the result. 
We required the log probability and the gradient of the log
probability to reach machine precision before considering the optimisation
converged. After convergence, we queried whether any points within the ball of
similar stars had not been optimized. If no result existed, we would next move to that
point. We refer to this process as \emph{swarm optimisation}, which we conducted in 
parallel, and illustrate in Figure \ref{fig:swarm_optimisation}.

\todo{To minimise computational cost we assigned the closest 8 stars with the result
from others}

\subsubsection*{Step 3: Evaluating probability and system properties}

 
\todo{Bayes factors for each dimension?}

\todo{source properties: semi-amplitude (easy)}

\todo{Can we recover period and eccentricity using The Joker?}




%For higher-order star systems with longer periods, we can still find that a
%stellar multiple system is more likely than a single star scenario, but our
%estimate of the orbital properties (e.g., the semi-amplitude $K$) will be biased
%to lower values because we have not fully sampled half of the orbital period.
%At even longer orbital periods, or for stellar multiples with mass ratios that
%produce a small velocity semi-amplitude $K$, under our assumptions these systems 
%may be classified as single-star systems because the intrinsic radial velocity variation is within the expected 
%variations for single stars of a similar apparent magnitude and colour. We 
%revisit this issue in Section \ref{sec:sb_limits}, where we explicitly define 
%the probability that a \Gaia\ source is a SB1-type system $p(\textrm{SB1}|y)$ 
%within defined limits of orbital period (and other orbital and source parameters). 
%Outside of this parameter range, we are insensitive to distinguishing single-star 
%systems from higher-order SB1-type systems.





%\subsection{Unresolved near-equal mass binaries: photometry}
%\label{sec:pb_methods}

%If a binary system is observed nearly face-on (at an inclination angle 
%$i \approx 0^\circ$) then there is will be no detectable excess radial 
%velocity variations. In principle there may be detectable astrometric
%variations, but most near-equal mass binaries would be detected more 
%reliably through intrinsic magnitudes that are anomalously lower (brighter)
%and bluer colours than what would be expected for a single star system.




\subsection{SB2 Systems: Double-lined spectroscopic binaries}
\label{sec:sb2_methods}

Many sources in the second \Gaia\ data release do not have a reported radial 
velocity, despite being bright enough ($G \lesssim 13$) and in a suitable 
temperature range (e.g., between $\approx4000\,\textrm{K}$ and $\approx6500\,\textrm{K}$) 
for radial velocities to be measured \citep{Cropper:2018}. The principle reason 
why radial velocity measurements are \emph{not} reported for these stars is 
because the \Gaia/DPAC team have identified the source to be a double-lined 
spectroscopic binary (a so-called SB2-type system), either through a 
composition of two stellar sources present in the spectra, or from multiple 
(significant) modes in the cross-correlation function. In these situations it
is not sensible to report a point estimate of the radial velocity of the point 
source. For fainter or bluer sources, however, the radial velocity 
may not be reported simply because it is too faint, blue, or red. 


We calculated the completeness of radial velocity measurements (e.g., the
fraction of sources with reported radial velocities) as a function of all 
available source properties that might affect whether the radial velocity may
be reported or not. This included position ($\alpha$, $\delta$, $l$, $b$),
parallax, proper motions, apparent magnitudes, \bp\ - \rp\ colour, 
properties of the radial velocity templates, stellar parameters ($\teff$, $\radius$, $\luminosity$),
and other properties. The full list of properties investigated is described in
Appendix \ref{appendix:sb2}, with corresponding figures. 

In Figure \ref{fig:radial_velocity_completeness} we show the completeness as 
a function of some pertinent properties, which demonstrate that the radial 
velocity completeness is relatively flat until a source becomes too faint 
(low \rp\ flux), or is either too blue or too red. We adopt conservative 
limits for when the radial velocity completeness starts to drop with these 
properties, and we assume that any source within the following range of 
source parameters
\begin{eqnarray}
    \mathrm{phot\,\,rp\,\,mean\,\,flux} & \in & [10^{4}, 10^8] \nonumber \\
    \mathrm{bp-rp} & \in & [X, Y] 
    \label{eq:sb2_criteria}
\end{eqnarray}
\noindent{}is likely to be a double-lined spectroscopic binary (SB2) if no 
radial velocity is reported. That is to say that we are explicitly assuming
that if a point source meets the criteria in Equation \ref{eq:sb2_criteria} 
and does not have a reported radial velocity measurement, then the point 
source is an unresolved double-lined spectroscopic binary. In principle we 
could make more realistic attempts to model the radial velocity completeness as
a function of stellar properties rather than simply stating ``sources within
this parameter range should have radial velocities unless they are SB2 systems'',
but the radial velocity completeness within our specified range is reasonably
flat.


\todo{Some confirmation of this? The properties of SB2 candidate systems relative to others. Figure \ref{fig:sb2_histograms}.}


We stress that our sample of SB2 type systems is calculated (or rather, deduced)
separately to our calculations of multiplicity probability given the 
\emph{measured} quantities from \Gaia\ like radial velocity, astrometry, and
photometry. In Section \ref{sec:discussion} we detail some of the immediate
limitations or interpretations of this assumption, and we emphasise that the 
main conclusions of this work do not depend on SB2 systems. Nevertheless, we 
\emph{strongly} 
caution that our deductive inference of SB2 systems will be far more uncertain 
than the binary probabilities we derive from other information. Our indicator 
whether a star is likely a SB2 system likely suffers from more contamination
and lower completeness compared to the probabilities we derive from other information
(e.g., the radial velocity jitter, astrometric excess noise, and photometric 
colours), and the contamination and completeness likely vary over some
complex (unknown) function of parameter space. 




\begin{figure*}
%	\includegraphics[width=1.0\textwidth]{../figures/sb2_rvs_completeness.pdf}
	\includegraphics[width=1.0\textwidth]{../figures/todo.png}
    \caption{Fraction of \Gaia\ sources with reported radial velocities
		     as a function of source properties. Within the adopted source
		     parameter ranges (gray; see Section \ref{sec:sb2_methods})
		     the radial velocity completeness is approximately flat.
		     We flag sources within this range that do not have reported 
		     radial velocities to be candidate SB2 systems.}
    \label{fig:sb2_rvs_completeness}
\end{figure*}




\begin{figure}
%	\includegraphics[width=0.4\textwidth]{../figures/sb2_histograms.pdf}
	\includegraphics[width=0.4\textwidth]{../figures/todo.png}
    \caption{Distribution of 
    			(a) astrometric unit weight error (see Eq. \ref{eq:astrometric_unit_weight_error}),
			%$u = \left(\chi_{al}^2/(N_{al} - 5)\right)^{1/2}$
			%		where $\chi_{al}^2$ and $N_{al}$ is the astrometric goodness-of-fit and number 
			%		of observations in the along-scan direction, respectively (columns 
			%			\texttt{astrometric\_chi2\_al} and \texttt{astrometric\_n\_obs\_al}), 
				(b) photometric \bp\ - \rp\ excess factor, 				
	 				and 
				(c) photometric \rp\ variability (see Eq. \ref{eq:photometric_variability})
			of candidate SB2 type systems relative to a control sample of the same size.}
    \label{fig:sb2_histograms}
\end{figure}



\begin{figure*}
%	\includegraphics[width=1.0\textwidth]{../figures/sb2_sky_structure.pdf}
	\includegraphics[width=1.0\textwidth]{../figures/todo.png}

    \caption{Fraction of sources (per sky bin) without reported radial velocities
    		 for all sources with $G \lesssim 13$ (top) and sources in the ranges
		 	 specified by Eqs \ref{eq:sb2_ranges} (bottom), where we assert that a missing
			 radial velocity measurement signifies a likely SB2 candidate . The color scale is 
			 arbitrarily set to highlight structure, where black indicates a higher
			 fraction of sources do not have reported radial velocities. Crowding
			 in the galactic plane likely results in some sources not having radial
			 velocities reported. The large scale structure visible in both axes is
			 a combined effect of the initial \Gaia\ source list, the scanning law,
			 and star forming regions (i.e., where emission in the \ion{Ca}{2} 
			 triplet likely causes issues for radial velocity determination).}
    \label{fig:sb2_sky_structure}
\end{figure*}


% Figure: cross-match against a catalog of SB2 systems,... do we find all of
% 		  what they find from SB2 deductive inference alone? or do we miss
% 		  some SB2s in some parameter range?


%\subsection{Photometric binaries} \label{sec:method_photometric_binaries}

%\begin{eqnarray}
%	\textrm{Amp} = \log_{10}\left(\sqrt{N_{obs,al}}\frac{\sigma_{\langle{}f_G\rangle}}{\langle{}f_G\rangle}\right) \label{eq:photometric_variability}
%\end{eqnarray}




\section{Results} \label{sec:results}

\section{Discussion} \label{sec:discussion}


\begin{itemize}
	\item \todo{Binary fraction in the spaces that we were fitting: rp flux, colour, etc, just showing the transition of  probabilities}
	\item \todo{Binary fraction as a function of fitting properties (e.g., colour, absolute RP mag, apparent RP mag)}
	\item \todo{Binarity across the H-R diagram}
	\item \todo{What are the distributions of orbital parameters of binary systems that we would be able to detect?}
\end{itemize}

\subsection{Comparison with \citet{El-Badry:2018}}

\subsection{Comparison with \citet{Badenes:2018}}

\subsection{Comparison with \citet{Raghavan:2010}}

\subsection{Comparison with \citet{Troup:2016}}

\subsection{Comparison with \citet{Price-Whelan:2018}}

\begin{itemize}
	\item \todo{Binarity with metallicity }
    \item \todo{binarity in clusters vs the field?}
    \item \todo{binarity among extremely metal-poor stars?}
\end{itemize}


\acknowledgements

% At least some of these people will be promoted to the author list.
It is a pleasure to thank
	Berry Holl (Observatoire de Gen\'eve),
	Jose Hernandez (ESAC),
	Daniel Michalik (ESA/ESTEC),
	Kevin C. Schlaufman (Johns Hopkins University),
	Lorenzo Spina (Monash University),
		and
	Sergey Koposov (Carnegie Mellon University).
This work was supported in part by the Australian Research Council 
through Discovery Grant DP160100637.
This work has made use of data from the European Space Agency (ESA) mission {\it
Gaia} (\url{https://www.cosmos.esa.int/gaia}), processed by the {\it Gaia} Data
Processing and Analysis Consortium (DPAC,
\url{https://www.cosmos.esa.int/web/gaia/dpac/consortium}). Funding for the DPAC
has been provided by national institutions, in particular the institutions
participating in the {\it Gaia} Multilateral Agreement.  This research was
developed in part at the NYC Gaia DR2 Workshop at the Center for Computational
Astrophysics of the Flatiron Institute in 2018 April.

This work has made use of CosmoHub. CosmoHub has been developed by the Port 
d'Informaci\'o Cient\'ifica (PIC), maintained through a collaboration of the 
Institut de F\'isica d'Altes Energies (IFAE) and the Centro de Investigaciones 
Energ\'eticas, Medioambientales y Tecnol\'ogicas (CIEMAT), and was partially 
funded by the ``Plan Estatal de Investigaci\'on Cient�fica y T\'ecnica y de 
Innovaci\'on'' program of the Spanish government.


\appendix
\section{Reproducibility}
This project was developed in a \texttt{git} repository that is publicly
accessible at \giturl. The repository includes notebooks that detail our work,  
\LaTeX\ to compile this manuscript, and scripts to reproduce the results presented in
this manuscript. The results presented here can be reproduced in full (including data
retrieval, analysis, creation of figures, and manuscript compilation) using these
commands on a modern terminal:
% this is not python, but latex fails to evaluate \githash if I set bash environment.
\begin{minted}[
style=friendly,
escapeinside=||
]{python} 
git clone https://github.com/andycasey/velociraptor.git velociraptor
cd velociraptor
git checkout |\githash|
./reproduce
\end{minted}

Reproducing these results will require at least \todo{X}\,Gb of free disk space 
and \todo{Y}\,hours of compute time. The settings in \texttt{model.yaml} can be 
adjusted to reduce the sample size of the data and shorten the compute time, but
varying these settings does not guarantee exact replication of the results shown
here.


\section{Radial velocity completeness as a function of \Gaia\ source properties}
\todo{Figures}

% astroquery
% minted
\software{
	\package{Astropy} \citep{astropy:v1,astropy:v2},
    \package{IPython} \citep{ipython},
    \package{matplotlib} \citep{mpl},
    \package{numpy} \citep{numpy},
    \package{scipy} \citep{scipy},
    \package{Stan} \citep{stan},
    \package{CosmoHub} \citep{cosmohub},
    \package{TensorFlow} \citep{tensorflow}
    \package{Jupyter Notebooks} \citep{jupyter-notebooks}
}    

\bibliographystyle{aasjournal}
\bibliography{velociraptor}



\end{document}
