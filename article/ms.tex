\documentclass[twocolumn]{aastex61}
\usepackage{bm}

\newcommand\teff{T_{\rm eff}}
\newcommand\logg{\log{g}}
\newcommand\feh{[\rm{Fe}/\rm{H}]}
\newcommand\mh{[\rm{M}/\rm{H}]}

\newcommand{\project}[1]{\textsl{#1}}
\newcommand{\acronym}[1]{{\small{#1}}}
\newcommand{\Gaia}{\project{Gaia}}
\newcommand{\gaia}{\project{gaia}}

\received{2018 XX XX}
\revised{2018 XX XX}
\accepted{2018 XX XX}

%\submitjournal{AAS Journals}

\shorttitle{Stellar multiplicity}
\shortauthors{Casey et al.?}

\begin{document}

\title{Stellar multiplicity}

\correspondingauthor{Andrew R. Casey}
\email{andrew.casey@monash.edu}

\author[0000-0003-0174-0564]{Andrew R. Casey}
\affiliation{School of Physics \& Astronomy, Monash University, Clayton 3800, 
    Victoria, Australia}
\affiliation{Faculty of Information Technology, Monash University, Clayton 3800, 
    Victoria, Australia}

%\author[0000-0003-2866-9403]{David W. Hogg??}
%\affiliation{Center for Cosmology and Particle Physics, Department of Physics,
%    New York University, 726 Broadway, New York, NY 10003, USA}
%\affiliation{Center for Data Science, New York University, 60 Fifth Ave, 
%    New York, NY 10011, USA}
%\affiliation{Max-Planck-Institut f\"ur Astronomie, K\"onigstuhl 17, D-69117 
%    Heidelberg}
%\affiliation{Flatiron Institute, 162 Fifth Ave, New York, NY 10010, USA}

\begin{abstract}
Velociraptors can't see you if you don't move. (\Gaia can see you if you can.)
\end{abstract}


\keywords{stars: dynamics}

\section{Introduction} \label{sec:intro}

\begin{itemize}
\item the binary fraciton is important
\item Gaia provides exquisite astrometry and RV measurements over many years
\item The errors in these quantities are well-characterised
\item When a star is in a binary system (or triary, etc), there will be excess scatter relative to a single star of the primary's properties
\item the excess RV noise (and in DR1 or DR3, excess astrometric noise) can be used to infer the presence of a companion
\end{itemize}


\section{Model}

\begin{itemize}
\item The effect of binaries.

\item RV noise vs astrometric excess noise.

\item When would we expect to see astrometric noise and not RV noise, and vice-versa? in what situations?

\item Here we base our model using RV data only from Gaia DR2.

\item Some fraction of stars in Gaia are single star systems (or have such low semi-amplitudes that would not be detected by Gaia), and we will use those stars (within a mixture model) to calibrate the radial velocity error floor

\item We will do this on individual epoch measurements

\item ASSUMPTIONS

\item MODEL

\item CALIBRATION SAMPLE (including ADQL query)

\item Fitting the model:

\item Comparisons of residuals with other stellar properties.

\item What is the RV error that Gaia can measure in a single epoch for a given magnitude vs colour, etc. 

\item What are the distributions of orbital parameters of binary systems that we would be able to detect?

\item Calculating a probability of binarity for every star, taking the non-measurements into account!

\item Completeness cuts,.... selection effects?

\end{itemize}

\section{Results}

\begin{itemize}

\item Binary fraction in the spaces that we were fitting: rp flux, colour, etc, just showing the transition of  probabilities

\item Binary fraction as a function of fitting properties, and other stellar properties from Gaia

\item Binarity across the H-R diagram

\item Binarity with metallicity 

\item Binarity at height above the plane?

\item Binarity in clusters vs the plane?
\end{itemize}

\section{Discussion}



\software{numpy, scipy, astropy, Stan}

\acknowledgements
- If these peeps are not co-authors: Hogg, Spina, Schlaufman.
- Gaia/DPAC
- Gaia archive?
- Gaia DR2 hack week at Flatiron
- ARC DP

%This work has made use of CosmoHub.

%CosmoHub has been developed by the Port d'Informaci� Cient�fica (PIC), maintained through a collaboration of the Institut de F�sica d'Altes Energies (IFAE) and the Centro de Investigaciones Energ�ticas, Medioambientales y Tecnol�gicas (CIEMAT), and was partially funded by the "Plan Estatal de Investigaci�n Cient�fica y T�cnica y de Innovaci�n" program of the Spanish government.


\end{document}
